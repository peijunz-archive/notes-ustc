\documentclass[12pt]{article}
\usepackage{zpj}
\geometry{a4paper}%screen用于屏幕
\newcommand{\mline}{\underline{\hspace{4em}}}
\newcommand{\abcdf}[4]{(A)#1 \hfill (B)#2\hfill (C)#3 \hfill (D)#4\hfill}
\newcommand{\abcdt}[4]{\makebox[6.4cm][l]{(A)#1}(B)#2

\makebox[6.4cm][l]{(C)#3}(D)#4}
\newcommand{\dist}{\hspace{1.5em}}
\begin{document}
\title{2012--2013第一学期概率论期末考试试卷}
\author{}
\date{}
\maketitle
一.判断选择题(每题3分,答题请写在试卷上):

1.设$A$,$B$,$C$是三个随机事件, 则在下列不正确的是\mline .

(A)$A\cup (B\cap C)=(A\cup B)\cap (A\cup C)$
 
 (B)$(A\cup B)\cap C=A\cup (B\cap C)$
 
 (C)$A\cap (B\cap C)=(A\cap B)\cap C$
 
 (D)$A\cap (\overline{B\cap C})=(A\cap \bar B)\cup (A\cap \bar C)$

2.设事件$A$与自身独立, 则$A$的概率为\mline .

\abcdf{0}{1}{0或1}{1/2}

3.设$f(x)$和$g(x)$为两个概率密度函数, 则下述还是密度函数的是\mline .

\abcdt{$f(x)/g(x)$}{$f(x)−g(x)$}{$(f(x)+g(x))/2$}{$(1+f(x))(1−g(x))$}

4.随机变量$X$和$Y$独立,$Y$和$Z$独立, 且都有期望方差,则必有\mline .

\abcdt{$X$和$Z$独立}{$X$和$Z$不相关}{$X$和$Z$相关 }{Cov$(X,Y)=0$}

5.设$0<P(B)<1$, 则$P(A|B)=P(A|\bar B)$成立的充分必要条件是\mline .

\abcdt{$P(AB)=P(A)P(B)$}{$P(A+B)=P(A)+P(B)$}{$P(A)=P(B)$}{$P(A)=P(\bar B)$}

6.设$X_1, \ldots, X_n$为来自均匀分布$U(−\theta,\theta)$的一组样本,$\theta$为未知参数, 则下述量为统计量的是\mline .

\abcdt{$\bar X−\theta$}{$\max\limits_{1\leq i\leq n}(X_i−\theta )−\min\limits_{1\leq i\leq n}(X_i−\theta )$}{$\max\limits_{1\leq i\leq n}(X_i−\theta )$}{$\min\limits_{1\leq i\leq n}(X_i−\theta )$}

7.假设从两个独立的正态总体中各得到样本量为10的两组样本, 若两个总体的方差相同, 则使用两样本$t$检验时$t$分布的自由度为\mline .

\abcdf{9}{10}{18}{20}

8.当原假设$H_0$为真时,检验$\phi$有可能\mline .

\abcdt{犯一类错误}{犯第二类错误}{犯第一类或者第二类错误}{同时犯第一类和第二类错误}

9.假设总体密度为$f_\theta(x)$, 其中$\theta$为参数. 若$X$为来自该总体的样本, 则下述不正确的是\mline .

\abcdt{固定$x$时$f_\theta(x)$为似然函数 }{固定$\theta$时$f_\theta(x)$为似然函数}{固定$\theta$时$f_\theta(x)$为密度函数}{$f_\theta(x)$衡量了不同$\theta$下观测到$x$的可能性大小}

10.假设总体$X$为取值0,1,2的离散型随机变量, 且取各值的概率分别为$P(X=0)=0.5$,$P(X=1)=p$,$P(X=2)=0.5-p$, 其中$0<p<0.5$为参数. 则当使用拟合优度检验时,检验统计量的渐近卡方分布的自由度为\mline .

\abcdf{3}{2}{1}{0}

二.(15分)设昆虫产卵数目服从参数为1的Poisson分布,而每个卵孵化为幼虫的概率为$p$,各卵是否孵化相互独立,
试求


(1)一个昆虫产生$m$个幼虫的概率。

(2)若已知某个昆虫产生了$m$个幼虫,求该昆虫产了$n (n\ge m)$个卵的概率。



三.(15分)设随机变量$X$,$Y$相互独立, 且$X$服从均匀分布$U(-1, 1)$,$Y$服从均值为$1/2$的指数分布, 则

(1)求随机变量$Z=(X+1)Y$和$X$的相关系数.

(2)求条件概率$P(Z> 1|X=0)$.


四.(15分)当PM2.5值全天监测平均在35微克/立方米以内时,空气质量属于一级. 现观测到合肥市琥珀山庄过去10天的日平均PM2.5值
分别为28.24, 31.48, 33.85, 39.34, 37.78, 30.21, 29.92, 31.21, 30.17, 37.84. 若假设琥珀山庄区域日均PM2.5值$X$服从正态分布,
各天日均PM2.5值相互独立. 则

(1)试给出日均PM2.5值的$95\%$置信上限.

(2)若感兴趣空气质量为一级的概率$p=P(X\le 35)$, 试基于观测的日均数据给出$p$的极大似然估计.


五.(15分)设甲乙两家食用盐工厂生产的食盐每袋重量均服从正态分布(忽略重量不可取负值). 现从这两家工厂产品中各随机抽出10件标称为500克
的袋装食盐, 分别测得抽出各袋食盐的重量(单位为克)为

甲厂: 495, 494, 500, 502, 501, 492, 495, 495, 499, 503;

乙厂: 494, 506, 496, 505, 500, 508, 502, 504, 502, 499.

试问甲乙两家工厂生产这种标称为500克的袋装盐重量上有无差异$(\alpha=0.05)$.


六.(10分)为研究人们每天阅读电子书的时间(T)长短与购买实体书(Y)两者之间的关系, 随机调查了210个人, 结果如下
\begin{table}[htbp!]
\centering
\begin{tabular}{|c|ccc|}
\hline &$t<1$&$1<t<3$&$t>3$\\\hline 购买 & 12 & 70 & 20 \\ 不购买 & 40 & 28 & 40 \\\hline 
\end{tabular}
\end{table}


试在水平$\alpha=0.05$下判断每天阅读电子书的
时间长短和购买实体书两者之间是否有关? 阅读电子书的时间长短和购买实体书之间呈现何种特点?
\newpage
一. 判断选择题(每题3分):

\begin{inparaenum}
\item B\hfill
\item C\hfill
\item C\hfill
\item D\hfill
\item A\hfill
\item B\hfill
\item C\hfill
\item A\hfill
\item B\hfill
\item C\hfill
\end{inparaenum}

二.(15分)
(1)
\begin{align*}
P(X=m)&=\sum_{n=m}^\infty P(X=m|Y=n)P(Y=n)=\sum_{n=m}^\infty\binom{n}{m}p^mq^{n-m} \frac{1}{n!}e^{-1}\\
&=\frac{p^m}{m!}e^{-p},\quad m=0,1,2,\ldots
\end{align*}

(2)
\begin{align*}
P(Y=n|X=m)&=\frac{P(X=m|Y=n)P(Y=n)}{P(X=m)} \\
&=\frac{q^{n-m}}{(n-m)!}e^{-q},\quad n=m,m+1,\ldots
\end{align*}

三.(15分)
(1)由于$EZX=E[X(X+1)]EY=1/6$,$Var(Z)=5/12$, 因此
$$\rho_{Z,X}=[EZX-EZ\cdot EX]/\sqrt{Var(Z)Var(X)}=1/\sqrt{5}$$
求随机变量$Z=(X+1)Y$的期望和方差.

(2) $P(Z> 1|X=0)=P(Y>1)=e^{-2}$.

四.(15分)(1)由于$\bar{x}=33.004, s=3.95$, 在题设下易知日均PM2.5的 95\%置信上限为$\bar{x}+\frac{S}{\sqrt{n}}t_{0.05}(n-1)$, 带入样本值得到35.5.

(2) $p=P(X\le 35)=P(\frac{x-\mu}{\sigma}\le \frac{35-\mu}{\sigma})=\Phi(\frac{35-\mu}{\sigma})$, 而$\bar{x}$和$\sqrt{(n-1)/ns^2}=\sqrt{9/10}s=3.75$为$\mu$和$\sigma$的似然估计值,因此
 $p$的极大似然估计值为$\hat{p}=\Phi(\frac{\bar{x}-\hat\mu}{\hat\sigma})=\Phi(0.53)=0.70$.

五.(15分)记$N(\mu_1,\sigma_1^2)$和$N(\mu_1,\sigma_1^2)$分别表示两家工厂袋装盐的重量分布

(1)考虑方差是否一致: 对假设$H_0:\sigma_1^2=\sigma_2^2\leftrightarrow H_1:\sigma_1^2\ne\sigma_2^2$, 由
 检验统计量$F=S^2_x/S_y^2=14.71/19.6=0.75> F_{0.975}(9,9)=1/F_{0.025}(9,9)=1/4.03=0.25$, 因此在0.05水平下不能拒绝零假设。

(2)考虑均值是否一致:考虑假设$H_0:\mu_1=\mu_2\leftrightarrow H_1:\mu_1\ne\mu_2$, 由(1)结果知可以使用两样本$t$检验,由检验统计量$T=|\bar{x}-\bar{y}|/\sqrt{(s^2_x+s^2_y)/10}
 =2.16>t_{0.025}(18)=2.10$, 因此拒绝零假设。即在0.05水平下拒绝“两家工厂的袋装食盐平均重量一致”这一假设。

六.(10分)假设每天阅读电子书时间长短与购买实体书之间无关,则由Pearson卡方检验有
 $T=\sum\frac{(O-E)^2}{E}=39.60>\chi_{0.05}(2)=5.99$,因此在0.05水平下拒绝“每天阅读电子书时间长短与购买实体书之间无关”这一假设。注意到在三类阅读时间下,购买实体书人的比例分别为0.23,0.71和0.33, 因此每天阅读电子书时间在1小时和3小时之间的人群购买实体书的比例最高,而当每天阅读电子书时间长于3小时后, 购买实体书的人比例反而下降为0.33.
\end{document}