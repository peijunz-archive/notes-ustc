\documentclass[a4paper,12pt]{ctexart}
\usepackage[left=1in, right=1in, top=1.5in, bottom=1.5in]{geometry}
\usepackage{zpj}
\usepackage{empheq}
\usepackage{qrcode}
% \title{\qrcode[hyperlink,height=1in]{https://github.com/peijunz/mathphys/releases}\hspace{1cm}第二次作业参考\thanks{下载地址: \url{https://github.com/peijunz/mathphys/releases},或直接扫描二维码}}
\title{作业参考解答}
\author{章鹤梓}
\newcommand{\num}[1]{\noindent\fcolorbox{RoyalBlue}{SkyBlue}{#1}~}
\newcommand{\emp}[1]{{\heiti #1}}
\begin{document}
\zihao{-4}
\maketitle
\pagestyle{plain}
\section{第二次作业习题解答}
\num{1}(1) $\pp u/\pp x=f'(xy)\pp(xy)/\pp x=yf'$,同理$\pp u/\pp y=xf'$。易得结论。

\num{12}(1) 特征方程:
\begin{equation}
\frac{\dd x}{y+z}=\frac{\dd y}{z+x}=\frac{\dd z}{x+y}
\end{equation}
由合分比定理易得出
\begin{equation}
\frac{\dd(x-y)}{x-y}=\frac{\dd(y-z)}{y-z}\quad\Rightarrow \dd\big(\ln(x-y)-\ln(y-z)\big)=0
\end{equation}
因此$\dfrac{x-y}{y-z}$是运动积分,同理$\dfrac{z-x}{y-z}$也是积分常数。因此有通解
\begin{equation}u=u\left(\dfrac{x-y}{y-z},\dfrac{z-x}{y-z}\right)
\end{equation}

\num{13}(1) 特征方程:
\begin{equation}
\dd x_i/\sqrt{x_i}=2\dd\sqrt{x_i}=2\dd\sqrt{x_j}
\end{equation}
故对任意$i\neq j$而言$\sqrt{x_i}-\sqrt{x_j}$是积分常数。因此有通解
\begin{equation}
u=u(\sqrt x-\sqrt y, \sqrt x-\sqrt z)
\end{equation}
代入边值得$u(1-\sqrt y, 1-\sqrt z)=y-z$,因此$u(m,n)=(m-1)^2-(n-1)^2$
\begin{align}
u(\sqrt x-\sqrt y, \sqrt x-\sqrt z)&=(\sqrt x-\sqrt y-1)^2-(\sqrt x-\sqrt z-1)^2\\
&=(\sqrt z-\sqrt y)(2\sqrt x-\sqrt y-\sqrt z-2)\\
&=y-z+2(\sqrt x-1)(\sqrt z-\sqrt y)
\end{align}

\num{14}(1) 积分一次得$u=x^2t+f(x)$,代入边值得$f=x^2$故$u=x^2(1+t)$

(2)易得$u=u(at+x)$,代入边值得$u(x)=x^2$故$u=(at+x)^2$

\num{略难的作业} \emp{题目中的光滑不需要分析的过于复杂,只需要分析连续性即可,可微性等不做要求}

(i) 由特征方程$\dd x/y=-\dd y/x$知有首次积分$r=\sqrt{x^2+y^2}$,设另一个变量$\phi=\mathrm{Arg}(x+y\ii)$,则有:
\begin{equation}
\left\{
\begin{aligned}
x&=r\cos\phi\\
y&=r\sin\phi
\end{aligned}
\right.\quad
\Rightarrow \pp/\pp \phi=
\frac{\pp x}{\pp \phi}\frac{\pp}{\pp x}+\frac{\pp y}{\pp \phi}\frac{\pp}{\pp y}=-y\frac{\pp}{\pp x}+x\frac{\pp}{\pp y}
\end{equation}

因此,若$c\neq 0$,原方程化为$\pp_\phi u-c(u-f/c)=0$,$u$有通解%设$v=u-f/c$,则对应$v$通解为
\begin{equation}
u(r,\phi)=g(r)\exp(c\phi)+f/c,\quad c\neq 0
\end{equation}
$g(r)$为任意函数。

否则当$c=0$时,方程化为
\begin{align}
 \pp_\phi u&=-f\newline\\
 u&=-f\phi+g(r)
\end{align}

在$r=0$处,$u$必须与$\phi$无关。得出
\begin{equation}
\left\{
\begin{aligned}
  g(0)&=0, \quad &c\neq 0\\
  f&=0, \quad &c=0
\end{aligned}\right.
\label{gzero}
\end{equation}

总之:
\begin{equation}\boxed{
\left\{
\begin{aligned}
 u(r,\phi)&=g(r)\exp(c\phi)+f/c,\; g(0)=0\quad &c\neq 0\\
 u(r,\phi)&=g(r)\quad &c=0
\end{aligned}\right.}
\end{equation}

\emp{对ii与iii的分析}~全平面内单值的解$u$要满足周期性边界条件$u(r,\phi)=u(r,\phi+2\pi)$。因此需要附加额外条件:
\begin{equation}
\left\{
\begin{aligned}
 \exp(2\pi c)&=1,\quad c\neq 0\\
 c&=0
\end{aligned}
\right.\quad\Rightarrow c=n\ii, n\in \mathbb{Z}
\end{equation}

(ii) 在$u$的解非平庸的前提下,为了让$u$是实函数,需要满足$c$为实数。又由于$c=n\ii$的限制,$c=0$,因此$f=0$。

(iii) 此时只需要$c=n\ii$即可,$f$可以是任意复数

(iv) 正负半轴分别对应$\phi=0, \phi=\pi$时的情况,$g(r)$可由任意一边确定下来:
\begin{equation}
g(r)=\theta(r)-f/c=(\theta(-r)-f/c)\exp(-c\pi),\quad r\geq 0
\end{equation}
前面的等号说明,第一问中$g(r)$需要满足的条件如\ref{gzero},$\theta(r)-f/c$也需要同时满足;后面的等号就是$\theta(x)$要满足的自洽性要求。
% \begin{equation}
% \Rightarrow\boxed{\left\{
% \begin{aligned}
%  \frac{\theta(-r)-f/c}{\theta(r)-f/c}&=\exp(c\pi),&\forall r>0\\
% \theta(0)-f/c&=0,&r=0
% \end{aligned}
% \right.}
% \end{equation}
\subsection{最后一个题目原点处可微性/光滑性的讨论}
这小节都是附加内容,不做要求。但是解微分方程采用极坐标/球坐标在原点处是需要经过一些额外讨论的,这是极坐标/球坐标的一种缺陷。前面已经利用原点处的连续性条件得出了\ref{gzero}。这里不讨论$c=0$的简单情形。
\subsubsection{一阶可微性}\label{1st}
原点处方向导数
\begin{equation}
\frac{\pp u}{\pp \bm n}=\lim_{r\to 0} \frac{u(r\cos\theta,r\sin\theta)}{r-0}=g'(0)\ee^{c\theta},\quad \bm n=(\cos\theta,\sin\theta)
\end{equation}
这里利用了$g(0)=0$的条件,而且转换到直角坐标$u=u(x,y)$。因此可以得出:
\begin{equation}
u_{x+}=g'(0), \quad u_{y+}=g'(0)\ee^{c\pi/2}
\end{equation}

可微要求
\begin{align}
 \frac{\pp u}{\pp \bm n}&=u_x\cos\theta +u_y\sin\theta \\
 g'(0)\ee^{c\theta}&=g'(0)(\cos\theta-\sin\theta\ee^{c\pi/2})
\end{align}
代入$c=n\ii$得到:
\begin{equation}
g'(0)(\ee^{\ii n\theta}-\cos\theta-\sin\theta\ee^{\ii n\pi/2})=0
\end{equation}
因此有两种可能:
\begin{equation}
\left\{
\begin{aligned}
 g'(0)&=0\\
 n&=\pm 1
\end{aligned}
\right.
\end{equation}
\subsubsection{光滑性}
一般情况下容易证明$u=z^n+f/c=r^n\exp(\ii n\phi)+f/c$是$c=\ii n$的时候的解$g=r^n$,而且此时$u$是解析的所以一定满足光滑性要求。但是如果要得到所有解,会有与$n$有关的限制,就像\ref{1st}这种特殊情况一样。

% ,$u(r,0)=\theta(r)\Rightarrow g(r)=\theta(r)-f/c$。对于任意的$(r,\phi)$,只要给定的$\theta(x)$误差足够小,总可以把误差控制下来,因此是稳定的。总而言之,这个解是适定的。

\end{document}
