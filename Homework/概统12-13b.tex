\documentclass[12pt]{article}
\usepackage{zpj}
\geometry{a4paper}%screen用于屏幕
\newcommand{\mline}{\underline{\hspace{4em}}}
\newcommand{\abcdf}[4]{(A) #1 \hfill (B) #2\hfill (C) #3 \hfill (D) #4\hfill}
\newcommand{\abcdt}[4]{\makebox[6.4cm][l]{(A) #1}(B) #2

\makebox[6.4cm][l]{(C) #3}(D) #4}
\newcommand{\abcdo}[4]{(A) #1 \par (B) #2\par (C) #3 \par (D) #4\par}
\begin{document}
\title{2012--2013第二学期概率论期末考试试卷}
\author{}
\date{}
\maketitle
一.判断选择题(每题3分,共30分,答题请写在试卷上):

1.设$A$,$B$,$C$为三个事件,则事件$\overline{ABC}$表示的是\mline .

\abcdt{$A$,$B$,$C$不同时发生}{$A$,$B$,$C$中至少发生一个}{$A$,$B$,$C$中至多发生一个}{$A$,$B$,$C$至少发生两个}

2.随机变量$X\sim N(\mu$,$\sigma^2)$,且关于$y$的一元二次方程$y^2+4y+X=0$无实根的概率为0.5,则$\mu=$\mline .

\abcdf{2}{3}{4}{5}

3.在数字1,2,3,4,5中不放回地随机连取两个数,每次一个数.则在第一次取出偶数的条件下,第二次取出奇数的概率为\mline .

 \abcdf{1/2}{3/4}{1/4}{1/3}

4.设随机变量$X_1$,$X_2$,$X_3$,$X_4$独立同分布于标准正态分布,令$T=a(2X_1-X_2)^2+b(X_3+X_4)^2$.则$(a$,$b)=$\mline 时候$T$服从自由度为2的卡方分布.

\abcdf{$(0$,$\frac{1}{2})$}{$(\frac{1}{5}$,$\frac{1}{2})$}{$(\frac{1}{5}$,$0)$}{$(1$,$\frac{1}{2})$}

5.设随机变量$X$,$Y$的方差均为$\sigma^2$,且两者的相关系数为$-0.5$,则使得
$Z=\pi X+(1-\pi)Y$的方差最小的$\pi$是\mline .

\abcdf{$\frac12$}{$\frac13$}{$\frac14$}{$\frac15$}

6.设$X_1$,$\ldots$,$X_n$为来自某个存在期望$\mu$和方差$\sigma^2$的总体的一组样本,$\bar{X}=\frac{1}{n}\sum\limits_{i=1}^nX_i$,$S_n^2=\frac{1}{n-1}\sum\limits_{i=1}^n(X_i-\bar{X})^2$分别为样本均值和样本方差,则下述错误的是\mline .

\abcdt{$\bar{X}$具有渐近正态性}{$S_n^2$为$\sigma^2$的无偏估计}{$\bar{X}$为$\mu$的无偏估计}{$\sqrt{n}(\bar{X}-\mu)/S_n$服从$t_{n-1}$分布}

7.设参数$\theta$的95\% 置信区间在某组样本值下为$[1.2$,$2.2]$,则下述正确的是\mline .

\abcdo{ 区间$[1.2$,$2.2]$包含$\theta$的概率为95\%}{ 区间$[1.2$,$2.2]$包含$\theta$的概率为5\%}{ 区间$[1.2$,$2.2]$要么包含$\theta$要么不包含$\theta$}{ 以上都不对}

8.关于假设检验中检验方法的一类和二类两种错误,下述错误的是\mline .

\abcdo{两类错误不可避免}{固定样本量时两类错误不可能同时很小}{有可能同时犯一类和二类错误}{限制第一类错误概率的原则是假设检验理论中的通用做法}

9.假设总体$X\sim N(\mu$,$1)$,其均值$\mu$的95\% 置信区间为$[0.22$,$1.10]$,则概率$P(X\le 0)$的95\%置信区间为\mline .

\abcdt{$[\Phi(0.22)$,$\Phi(1.10)]$}{$[1-\Phi(1.10)$,$1-\Phi(0.22)]$}{$[0.22$,$1.10]$}{$[0$,$\Phi(1.10)]$}

10. $X_1$,$\ldots$,$X_n$为来自正态总体$N(\mu$,$1)$的样本,假设检验问题$H_0:\mu=0\leftrightarrow H_1:\mu=1$的0.05水平检验为$\sqrt{n}\bar{X}>1.645$,若要求该检验犯二型错误的概率也不超过0.05,则样本量$n$至少为\mline .

  \abcdf{9}{10}{11}{12}


二.(15分)有甲乙两只口袋,甲袋中有5只白球和2只黑球,乙袋中有4只白球5只黑球.先从甲袋中任取两球放入乙袋,然后再从乙袋中任取一球.
试

(1)求从乙袋中取出的球为白球的概率.

(2)若已知从乙袋中取出的球为白球,求从甲袋中取的两只球中有白色球的概率。


三.(15分)设随机变量$Y$的密度函数为$f_Y(y)=4y^3I(0<y<1)$,
随机变量$X$在给定$Y=y\;(0<y<1)$时服从均匀分布$U(0$,$y)$.试

(1)求随机变量$X$的边际密度.

(2)求$X$和$Y$的相关系数.


四.(20分)假设总体$X$的概率分布为
$X_1$,$\ldots$,$X_n$为从该总体中抽取的一组简单样本,则
\begin{table}[htbp!]
\centering
\begin{tabular}{c|ccc}
\hline
$X$& 0 & 1 & 2 \\\hline
 $P$&$p$&$1-2p$&$p$\\\hline
\end{tabular}
\end{table}

(1)据此给出参数$p$的矩估计量$\hat p_1$和极大似然估计量${\hat p}_2$.

(2) $\hat p_1$和${\hat p}_2$是否为无偏估计? 何者更有效?

(3)若$n=100$,且一组样本值中统计发现其中等于0的有23个,等于1的有53个,等于2的有24个.在显著性水平$\alpha=0.05$下,利用${\hat p}_2$和拟合优度检验方法,我们能否认为“该组样本来自于总体$X$”?



五.(20分)假设某工厂产品的某个指标服从正态分布$N(\mu$,$\sigma^2)$,$\mu$,$\sigma^2$均未知.现从该厂某批产品中随机抽取了50件产品测得该指标值的平均值为89.70和样本标准差为1.09.据此

(1)能否认为该批产品该指标的平均值为90($\alpha=0.05$).

(2)能否认为该批产品该指标的标准差不超过1($\alpha=0.05$).

(3)给出该批产品此指标均值的95\%置信区间,并与(1)中假设检验结果比较,能得出什么结论?
\newpage
一.(30分, 每题3分)

\begin{inparaenum}
\item A\hfill
\item C\hfill
\item B\hfill
\item B\hfill
\item A\hfill
\item D\hfill
\item C\hfill
\item C\hfill
\item B\hfill
\item C\hfill
\end{inparaenum}

二.(15分)(1) $A=$从乙袋中取出白球, $B_i$分别表示从甲袋中取出两只球中有$i$个白球$(i=0,1,2)$,则由全概率公式有
\begin{align*}
P(A)&=P(A|B_0)P(B_0)+P(A|B_1)P(B_1)+P(A|B_2)P(B_2)\\
&=\frac{4}{11}\cdot\frac{2}{42}+\frac{5}{11}\cdot\frac{20}{42}
+\frac{6}{11}\cdot\frac{20}{42}\\
&=\frac{38}{77}\approx 0.494.
\end{align*}

(2)由Bayes公式有
\[
P(\bar B_0|A)=1-\frac{P(A|B_0)P(B_0)}{P(A)}=1-\frac{4/11\cdot 2/42}{38/77}=55/57\approx 0.965.
\]


三.(15分)
(1)由联合概率密度函数$f(x,y)=4y^2I(0<x<y<1)$易得
$$f_X(x)=\frac43(1-x^3)I(0<x<1)$$

(2)易得$EX=2/5, Var(X)=14/225$ 以及$EY=4/5, Var(Y)=2/75$, $EXY=1/3$.
从而$\rho_{XY}=\frac{1/3-8/25}{\sqrt{14/225\cdot2/75}}=\sqrt{\frac{3}{28}}\approx 0.327$.




四.(20分) (1) $\hat p_1=\frac{a_2-1}{2}$和极大似然估计量${\hat p}_2=\frac{n-n_1}{2n}$.其中$a_2=\frac{1}{n}\sum_{i=1}^nX_i^2$, $n_1$为样本中等于1的个数.

(2) $E\hat p_1=p$和$E{\hat p}_2=p$均为无偏估计. $Var(\hat p_1)=\frac{5p-2p^2}{2n}> Var(\hat p_2)=\frac{p(1-2p)}{2n}$, 故似然估计更有效.

(3)卡方检验值为$0.0213<\chi^2_1(0.05)=3.841$, 从而不能拒绝该组样本来自总体$X$的原假设.



五. (20分)
该批产品该指标的平均值为89.70和样本标准差为1.09. 据此

(1)检验统计量$|\sqrt{n}(\bar X-90)/S|=1.946<t_{49}(0.025)=2.010$, 因此在0.05水平下不能拒绝该批产品该指标的平均值为90的原假设.

(2)检验统计量$(n-1)S^2=58.22<\chi^2_{49}(0.05)=66.339$, 因此在0.05水平下不能能否认该批产品该指标的标准差不超过1的原假设.

(3)置信区间为$[\bar X-\frac{S}{\sqrt{n}}t_{n-1}(0.025),\bar X+\frac{S}{\sqrt{n}}t_{n-1}(0.025)]=[89.39,90.01]$ 其包含了90这一点,因此在0.05水平下不能拒绝该批产品该指标的平均值为90的原假设. 该置信区间为检验检验问题(1)的接受域.
\end{document}
